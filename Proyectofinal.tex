\documentclass[10pt,a4paper]{article}
\usepackage[utf8]{inputenc}
\usepackage{amsmath}
\usepackage{amsfonts}
\usepackage{amssymb}
\author{martinez tovar juan daniel}
\title{proyectofinal.tex}
\begin{document}
	
\section{La influencia del uso de dispositivos m{\'o}viles dentro del entorno familiar.}


\section{Indice:}
	
	
-Planteamiento del problema 	

-Objetivos 	

-Hipotesis 	

-Justificaci{\'o}n 	

-Marcote{\'o}rico 	

-T{\'e}cnicas e instrumentos	

-Encuesta 	

-An{\'a}lisis de resultados 	

Correlaci{\'o}n	

-Conclusi{\'o}n 	

-Referencias 	


		
		\section{Planteamiento del problema:}
		
		
El uso de los dispositivos m{\'o}viles tiene como objetivo acercar a las personas que viven lejos , que est{\'a}n alejadas y una mejor y f{\'a}cil comunicaci{\'o}n a distancia , contrario a esto va acercando a los familiares distanciados y alejando a los que est{\'a}n cerca .En la actualidad el desarrollo de los m{\'o}viles ha dejado sus consecuencias en nuestros hogares, es natural que tomemos a estas tecnolog{\'i}as como algo positivo ya que nos ayuda a facilitar nuestras actividades, pero esta evoluci{\'o}n ha afectado en gran medida las relaciones familiares provocando principalmente distanciamiento entre padres e hijos. Este uso de los dispositivos no debe de repercutir en ninguna persona y tomar las cualidades de esta herramienta en pro de ella, teniendo conciencia que el uso adecuado  es tener una buena comunicaci{\'o}n y empat{\'i}a entre el usuario y el m{\'o}vil, no dejando a un lado el mundo exterior ni descuidando nuestros h{\'a}bitos. Para evitar que este problema siga creciendo debemos hacer conciencia en las personas para que enfrenten sus problemas, dejen de evitar la realidad y empiecen a compartir m{\'a}s momentos con sus seres queridos.


		
		
	\section{OBJETIVO GENERAL:} 
	
	
Reforzar los lazos entre familiares haciendo  conciencia en las personas ayud{\'a}ndolos a dejar de vivir a trav{\'e}s de la tecnolog{\'i}a, para que puedan enfrentar sus problemas, dejen de evitar la realidad y empiecen a compartir m{\'a}s momentos con sus seres queridos.	


	
\section{OBJETIVOS  PARTICULARES:} 



-	Establecer soluciones para recuperar costumbres familiares de socializaci{\'o}n.
-	Determinar el miembro de la familia m{\'a}s afectado por los dispositivos m{\'o}viles.

	\section{HIPOTESIS}
	
El uso de dispositivos m{\'o}viles facilita muchas de las actividades diarias, aunque al mismo tiempo ocasiona que los integrantes de la familia convivan menos ya que muchos de ellos prefieren estar utilizando el celular o alg{\'u}n otro dispositivo m{\'o}vil en lugar de convivir con la familia.
Al utilizar con mucha frecuencia los dispositivos el grado de convivencia entre los familiares disminuir{\'a} a tal grado que podr{\'i}a hacerse nula lo que ocasionara problemas en los diferentes miembros de cada familia.
El dispositivo m{\'o}vil mantiene en contacto a las personas con familiares que est{\'a}n lejos, pero aleja a los que se tienen cerca. 

	
	\section{JUSTIFICACI{\'O}N}
	
he elegido este tema porque hoy en d{\'i}a parece casi imposible sobrevivir sin tener un dispositivo m{\'o}vil con nosotros. Ya que al tratar sobre los dispositivos m{\'o}viles abarca diferentes contextos, como tambi{\'e}n los son los h{\'a}bitos culturales, valores y normas, ya que esto de alg{\'u}n modo se observa afectando  la vida comunicativa de todos los que pueden hacer usos de sus servicios y  tambi{\'e}n  de aquellos que no tienen la oportunidad de acceder a su uso, m{\'a}s sin embargo sin entrar en mucho  detalle los dispositivos m{\'o}viles se han vuelto  populares por todo y por todos, con el tiempo unos se va familiarizando  con la utilizaci{\'o}n de los dispositivos  m{\'o}viles los cuales ayudan que la comunicaci{\'o}n se expanda con el ofrecimiento de mantener en contacto a las personas en cualquier lugar y en cualquier momento, como obvio asiendo que esta acci{\'o}n tomo un papel  cada vez m{\'a}s importante en las relaciones interpersonales y en la comunicaci{\'o}n social, estando as{\'i} todo el tiempo disponible para cuando nos necesiten o por el hecho de acceder a la informaci{\'o}n y a los contactos que requiramos. El uso de los dispositivos m{\'o}viles tiene grandes ventajas, pero la proliferaci{\'o}n masiva de estos dispositivos generando nuevos tipos de problemas que no se ten{\'i}an en cuenta ni se imaginaban hace tiempo. 

Como ejemplo y bas{\'a}ndome de los resultados de las encuestas son:
-	El mal uso del tiempo libre
-	Falta de atenci{\'o}n en clases o trabajo.
-	Falta de dialogo entre compa{\~n}eros de escuela o trabajo.
-	Irresponsabilidad
Parece absurdo el hecho de pensar que los dispositivos moviles se inventaron por la raz{\'o}n de comunicar a las personas entre si y que en pocas palabras hoy en la actualidad sirva para todo lo contrario, para distraernos y alejarnos de las dem{\'a}s personas.
No tiene sentido el hecho de que los dispositivos sirvan para aislar a las personas que lo utilizan de la gente que los rodea.
Las reglas para utilizar el celular se marcaron en nuestra encuesta de forma literal seg{\'u}n las edades, pero cabe recordar que es importante analizar cu{\'a}les son los momentos o situaciones en los que su uso es adecuado.

	\section{MARCO TEORICO}
	
	En los {\'u}ltimos a{\~n}os el mundo ha experimentado avances tecnol{\'o}gicos que se producen a una velocidad sin precedentes, muchos de los cuales se ven reflejados en Internet y la telefon{\'i}a celular que, en muchos aspectos, resultan complementarios. 
El hombre ha tenido la necesidad de comunicarse con los dem{\'a}s desde tiempos remotos, a su vez, siempre ha buscado expresar pensamientos, ideas, emociones, de investigar, saber, obtener informaci{\'o}n creada, expresada y transmitida por otros. Cosa que lo ha influido hasta el d{\'i}a de hoy a crear medios para poder seguir comunic{\'a}ndose, a transmitir informaci{\'o}n, a decir lo que piensa de alguna u otra manera, estos medios los podemos ver como los dispositivos m{\'o}viles. 

El celular tiene grandes ventajas, pero la proliferaci{\'o}n masiva de estos aparatos ha generado nuevos tipos de problemas que nadie imaginaba hace algunos a{\~n}o, en el ambiente familiar como el mal uso del tiempo libre, la irresponsabilidad y en ambiente escolar la falta de atenci{\'o}n en las clases, no tener respeto a los docentes durante clases, como la falta de di{\'a}logo entre compa{\~n}eros.

En el pasado como hasta ahora en la actualidad, la nueva tecnolog{\'i}a a buscado una estrategia de aumentar sus ventas mediante el uso de los dispositivos m{\'o}viles con ofertas y promociones que ofrecen las empresas de comunicaciones, es una de las causas que provocan que los estudiantes tengan m{\'a}s facilidad al acceso de estos servicios, causando que nosotros como estudiantes le damos mal uso al celular convirti{\'e}ndolo en una. 

Son muchos los aspectos en los que estos dispositivos pueden innovar nuestra manera de ver la vida e interactuar con la sociedad, tanto a nivel personal como profesional, una de las posibilidades m{\'a}s evidentes y que pr{\'a}cticamente todo el mundo aprecia el uso de dispositivos m{\'o}viles por su funci{\'o}n de entretenimiento. Aunque parezca mentira, muchos de los usuarios de Smartphone no concebir{\'i}an su vida sin los dispositivos m{\'o}viles en general, ya sea un tel{\'e}fono inteligente o una tablet. Las tendencias  para m{\'o}vil como antes se menci{\'o}n, puede haber una gran variedad dependiendo la edad o {\'a}rea en la que la persona se encuentre,  mayormente en promedio su uso es de entretenimiento, como las de tratamiento y edici{\'o}n de im{\'a}genes, de m{\'u}sica o de cine, comunicaci{\'o}n, trabajo e incluso es en ocasiones m{\'a}s utilizado para ocio personal. La educaci{\'o}n y el aprendizaje ligado al uso de dispositivos m{\'o}viles es un tema del que ya os hemos hablado en varias ocasiones.

La creaci{\'o}n, b{\'u}squeda y obtenci{\'o}n de informaci{\'o}n son acciones esenciales y propias a la naturaleza humana, siendo la cultura el fen{\'o}meno macro por excelencia de la socializaci{\'o}n del conocimiento (Cornejo y Tapia, 2011).
El uso de dispositivos m{\'o}viles facilita muchas de las actividades diarias, aunque al mismo tiempo ocasiona que los integrantes de la familia convivan menos ya que muchos de ellos prefieren estar metidos en el celular o alg{\'u}n otro dispositivo m{\'o}vil en lugar de convivir con la familia.

Los dispositivos m{\'o}viles han llegado para facilitar la vida de las personas, aunque tambi{\'e}n provoca que las personas convivan menos, a tal punto de que 8 de cada 10 personas se han sentido ignoradas por alguna persona que est{\'a} en el dispositivo m{\'o}vil mientras convive con alguien m{\'a}s.

Un estudio realizado por el centro de opini{\'o}n publica de la universidad del valle de M{\'e}xico mostro que 78 por ciento de los encuestados acepto que los dispositivos m{\'o}viles ha ido cambiando la manera en que interact{\'u}an las familias, considerando que platican y conviven menos.

Hasta el d{\'i}a de hoy la creaci{\'o}n de dispositivos m{\'o}viles avanza cada vez m{\'a}s r{\'a}pido, siendo as{\'i},  que la dependencia hacia lo tecnol{\'o}gico ha ido en aumento a medida que nuestra sociedad va evolucionando hacia nuevas formas de comunicaci{\'o}n, informaci{\'o}n y entretenimiento.
Las facilidades que tenemos hoy en d{\'i}a en nuestras manos a trav{\'e}s de un simple dispositivo m{\'o}vil podr{\'i}an resultar realmente alarmantes hace unas d{\'e}cadas. Una de las funciones que m{\'a}s se emplean de estos medios son las llamadas, mensajes de voz y aplicaciones, adem{\'a}s de comprobar la hora.

De igual manera diversos estudios confirman que lo primero y {\'u}ltimo que ve a lo largo del d{\'i}a un porcentaje considerable de la poblaci{\'o}n es la pantalla de su tel{\'e}fono m{\'o}vil, ya sea por el hecho de que muchos de ellos lo emplean como simple despertador o para comprobar los nuevos mensajes acumulados de sus seres queridos.

No obstante, estas investigaciones no solamente se centran en los usuarios con m{\'o}viles que disponen de conexi{\'o}n a Internet, puesto que tambi{\'e}n, aquellos que carecen de dicha conexi{\'o}n,  hacen un uso frecuente del mismo.

Ahora m{\'a}s que nunca, las barreras que separaban unos pa{\'i}ses de otros se han difuminado, de manera que ya no hay l{\'i}mites para contactar con nuestros familiares o amigos, ni preocupaci{\'o}n por la distancia a la que se encuentren. 

Como as{\'i} se rompen barreras de distancia, las barreras personales aumentan, debido a que la facilidad que hay en comunicarnos crece y hoy en d{\'i}a ya no hacemos el mismo esfuerzo que hac{\'i}amos antes para poder comunicarnos y relacionarnos con las personas que se encuentran a nuestro alrededor.

Hoy en d{\'i}a la edad ya no es un factor, ya que tanto adultos y j{\'o}venes son personas que pasan horas en el m{\'o}vil, ya sea por diferentes usos en el m{\'o}vil: sin embargo, las relaciones a distancia van en aumento, m{\'a}s las relaciones personales siguen disminuyendo. 
Siendo as{\'i} como nuestra comunicaci{\'o}n va disminuyendo de alguna manera, por la comodidad de un dispositivo m{\'o}vil.

En {\'a}reas como escuelas es muy com{\'u}n observar como los ni{\~n}os aprenden con Tablet o dispositivos m{\'o}viles. Tal vez a un grado en el que los ni{\~n}os ya no aprenden si no viene de la mano de internet y los libros de texto, ya no son muy utilizados por ellos, ya que utilizan aplicaciones.
Actualmente, la tecnolog{\'i}a representa un gran avance en la modernidad, calificando su importancia como natural siempre y cuando se use de manera normal, es decir, sin exceso ni ansiedades; {\'u}nicamente para necesidades o asuntos de verdadera utilidad, como negocios, informar brevemente su ubicaci{\'o}n e indicar hacia donde se dirige, entre otras cosas. Sin embargo, la telefon{\'i}a m{\'o}vil, al igual que otras cosas como el trabajo, las compras, la televisi{\'o}n, etc. est{\'a}n provocando numerosos casos de dependencia entre los adolescentes, quienes encuentran en estas herramientas un refugio que los aleja de sus problemas emocionales o familiares; debido a lo anterior, las nuevas tecnolog{\'i}as han pasado a formar parte de las denominadas ``adicciones psicol{\'o}gicas o adicciones sin drogas", las cuales son conductas repetitivas que resultan placenteras en las primeras fases, pero que despu{\'e}s no pueden ser controladas por los usuarios.

Aunque no est{\'a} tipificado como ``adicci{\'o}n a Internet" en los manuales internacionales, ya podemos hablar de un ``Desorden Adictivo a Internet", cuando la conducta se repite por m{\'a}s de cuatro meses. Algunos indicadores son: uso compulsivo del medio, incapacidad de desconexi{\'o}n, cambios en ciclos biol{\'o}gicos, afectaci{\'o}n en otras {\'a}reas (escuela, relaciones, econom{\'i}a personal) y dejar de lado otras actividades. Como en otras adicciones hay factores que predisponen a las personas, como la dificultad para entablar relaciones interpersonales, el aislamiento o falta de contacto social, problemas personales, sentirse solo, baja autoestima, presiones y ambiente familiar violento. La adicci{\'o}n puede llevar a problemas f{\'i}sicos como la tendinitis (inflamaci{\'o}n de los tendones de la mano por el uso del teclado y el mouse), resequedad ocular, perdida en la agudeza visual, problemas vasculares por el poco movimiento, cambios en el ciclo de sue{\~n}o, de alimentaci{\'o}n, dejar de comer, etc. (El Universal, 2008)

  Estudios realizados en la Universidad de Straffordshire por el Doctor Davie Sheffield, revelaron que existen problemas de conducta relacionados con el uso de los tel{\'e}fonos m{\'o}viles. Asimismo, otro estudio revela que la presi{\'o}n sangu{\'i}nea de las personas que hab{\'i}an dejado de usar m{\'o}viles era m{\'a}s baja que quienes segu{\'i}an us{\'a}ndolos. Este problema es universal, seg{\'u}n indica en la BBC Ciencia el doctor Guillermo Lancelle, del Instituto Universitario de Salud Mental en Buenos Aires. Por otra parte, John O'Neill, director del servicio de adicciones de la Cl{\'i}nica Menninger de Houston (Texas), se refiere a la ``sobrecarga de tecnolog{\'i}a", cuando ve pacientes adictos al uso del tel{\'e}fono celular o del e-mail. Esta adicci{\'o}n comienza a verse cuando alguien no es capaz de abandonar el uso compulsivo de estos medios tecnol{\'o}gicos y al igual que las adicciones de alcohol drogas o juego, deterioran y destruyen los lazos sociales de la persona. (BBC Mundo, 2006)  

Los m{\'a}s vulnerables a esta adicci{\'o}n, son las personas j{\'o}venes, quienes desean tener siempre la {\'u}ltima versi{\'o}n tecnol{\'o}gica y lograr mejorar su status y autoestima, por lo que no pueden tener ratos de silencio o de soledad que les permitan pensar, hacer tareas cotidianas, dedicar un tiempo a la lectura o simplemente hacer otras actividades. En ocasiones, es tanta la dependencia que durante reuniones importantes, actos religiosos, proyecciones de cine, entre otras, las personas no pueden dejar de contestar el celular, por lo que se ha llegado a tomar la medida de solicitar a los asistentes que apaguen sus tel{\'e}fonos celulares para no interrumpir a los dem{\'a}s y permitir su misma concentraci{\'o}n.  

Las nuevas tecnolog{\'i}as amenazan tambi{\'e}n con da{\~n}ar las relaciones familiares. Es una de las recientes conclusiones de una reciente investigaci{\'o}n publicada por la revista Pediatrics, despu{\'e}s de hacer un seguimiento a 55 grupos familiares, se encontr{\'o} que casi el 75 por ciento de los casos, los adultos utilizan dispositivos m{\'o}viles durante la comida con sus hijos. En un 10 por cientos de los casos iba desde no sacar el tel{\'e}fono o ponerlo sobre la mesa y en 40 casos ocurri{\'o} que lo usaban de manera constante.

``Por las pantallas se corre el riesgo de acabar el di{\'a}logo en las familias, con el peligro que eso trae para la herencia, porque qu{\'e} se trasmite o qu{\'e} asimila un ni{\~n}o que tiene cortada la conversaci{\'o}n con el otro. Para {\'e}l pareciera que lo {\'u}nico que existen son im{\'a}genes, pero estas no tienen cuerpo, sustancia, ni trasmiten".

En medida que los padres centren la atenci{\'o}n en la tecnolog{\'i}a, van perdiendo la oportunidad de establecer contacto visual y de ver expresiones faciales que le ayudaran a comunicarle un mensaje de la vida de su hijo, esto puede generar en los ni{\~n}os sentimientos de inseguridad, de rabia, y la creencia de que no son importantes para los padres.

Los especialistas  se{\~n}alan que con esto se les da a entender a los ni{\~n}os que los momentos familiares no importan y no les ense{\~n}an a respetar la presencia del otro. Expertos plantean que los padres se encuentran en un conflicto pues quieren retirar los dispositivos m{\'o}viles a sus hijos para que no pasen tanto tiempo en ellos y por otro lado quieren tenerlos localizados. 

Con los Smartphones apareci{\'o} un gran problema, la hora de la comida que es la hora familiar y otros momentos familiares son interrumpidos por las vibraciones del celular adem{\'a}s de que afectan la comunicaci{\'o}n familiar. 

Supuestamente los dispositivos ayudan a la gente a estar m{\'a}s conectada, no siempre es as{\'i}. Los padres se quejan de la vida dicen que pasa tan r{\'a}pido que no tienen tiempo de convivir con sus hijos, pero en los momentos que pasan juntos , est{\'a}n conectados al celular para estar en contacto con otras personas.

Con las redes sociales sincronizadas a los celulares, el tel{\'e}fono se ha convertido en una v{\'i}a de escape para muchos ni{\~n}os en cualquier situaci{\'o}n. Pero no solo a los ni{\~n}os se debe de culpar ya que los padres tambi{\'e}n pasan mucho tiempo utilizando estos para responder una llamada o un mail muy r{\'a}pidamente.

El tel{\'e}fono m{\'o}vil ya no s{\'o}lo se usa para realizar llamadas o mandar mensajes. Hoy d{\'i}a, los tel{\'e}fonos de {\'u}ltima generaci{\'o}n junto con las tabletas digitales han adquirido m{\'u}ltiples usos, que muchas personas consideran que son ya indispensables para ellos. A trav{\'e}s de los dispositivos m{\'o}viles, las personas buscan informaci{\'o}n y realizan todo tipo de consultas como lo es: buscar la ruta m{\'a}s corta para llegar al destino, leer el peri{\'o}dico, consultar el horario de apertura de un comercio determinado, son algunas de las consultas m{\'a}s comunes. El {\'a}rea de la tecnolog{\'i}a m{\'o}vil ha tenido un especial crecimiento en los {\'u}ltimos a{\~n}os gracias a la aparici{\'o}n y al constante desarrollo de las tabletas y los Smartphone. Estos dispositivos nos ofrecen infinitas posibilidades, logrando responder a muchas de nuestras necesidades diarias.

La expansi{\'o}n de este fen{\'o}meno de dispositivos y el inter{\'e}s de ciertas empresas por ello, ha creado una nueva necesidad tanto para los usuarios como para los profesionales especializados en los dispositivos m{\'o}viles. ``No basta con estar presente en las redes sociales, o tener una p{\'a}gina web donde ofrecer informaci{\'o}n. Ahora hace falta desarrollar aplicaciones, conocer al p{\'u}blico objetivo, aumentar las funcionalidades de la web de una empresa, etc.".

Una caracter{\'i}stica importante cuando se utiliza un dispositivo m{\'o}vil, permite fundamentalmente aprovechar cualquier momento que tengamos libre para avanzar con la formaci{\'o}n, estar siempre atentos a novedades y permanentemente en contacto, de tal manera que las horas de desplazamientos, tiempo de esperas, se convierten en potenciales espacios para continuar form{\'a}ndonos, m{\'a}s que nada, no tener tantos obst{\'a}culos para tener informaci{\'o}n escasa y poder expandirnos a mas puertas dentro de un mundo de nueva informaci{\'o}n que se pueda encontrar, siempre y cuando se haga en buen uso de ello.

Al acceder con las herramientas m{\'o}viles se presentan mil formas de utilizarlos y en mejores t{\'e}rminos sociales para la interacci{\'o}n, se podr{\'i}a tomar de una manera que el usuario visualizar todos sus correos no le{\'i}dos, redactar nuevos correos, visualizar los contactos, participar en foros y trabajar con sus notas personales, siempre con la posibilidad de moverse por los distintos cursos en los que est{\'a} matriculado.

Cada persona le da un uso diferente y piensa diferente al utilizar m{\'o}viles, muchos por trabajo, otros por ocio, algunos m{\'a}s por comunicaci{\'o}n, son diferentes contextos los que abarca el uso del m{\'o}vil, pero en promedio, en la actualidad, las personas que utilizan dichos artefactos para comunicaci{\'o}n son las personas mayores a 25 a{\~n}os, por ocio, menores de 22 a{\~n}os y por trabajo mayores de 30 a{\~n}os, por lo que se podr{\'i}a interpretar que por cada etapa de la vida puede cambiar su uso. 

Parece dif{\'i}cil desarrollar el buen uso del celular y el manejo del tiempo para que tomen conciencia sobre la gravedad que puede generar la adicci{\'o}n al celular en los estudiantes durante las clases y en su tiempo libre durante su d{\'i}a a d{\'i}a y en la convivencia en su entorno familiar, y de igual forma de forma inversa los padres de familia de c{\'o}mo desarrollan el uso de los dispositivos m{\'o}viles en su trabajo o una ves estando en casa con su familia al momento de convivir con ella no interrumpa su lazo familiar el uso incorrecto o mayoritario de los dispositivos m{\'o}viles.

Se localizo el mayor problema del uso de dispositivos m{\'o}viles que es el no tener una buena comunicaci{\'o}n con las personas que los rodean, aunque esta herramienta sea {\'u}til debemos darle importancia a su buen uso en muchos aspectos de la vida cotidiana. Si nosotros como personas altamente consiente de nuestras necesidades y ocupaciones le di{\'e}ramos un buen uso a este medio de comunicaci{\'o}n podr{\'i}amos aprovechar todos los beneficios que este nos ofrece.

F{\'a}cilmente podemos observar cuando salimos a la calle de que no es raro ver a casi todos las personas con sus celulares, charlando o enviando mensajes. Seg{\'u}n las estad{\'i}sticas de los principales operadores de telefon{\'i}a celular, cada terminal env{\'i}a al d{\'i}a una media de 29 mensajes este dato nos puede dar una idea de la cantidad de dinero que mueven la telefon{\'i}a celular y de la realidad de la adicci{\'o}n a los celulares. La mayor{\'i}a de los adultos utilizan el celular de forma racional, para recados cortos, para acceder de manera r{\'a}pida a todo tipo de informaci{\'o}n, para estar comunicados mientras se desplazan, etc.

Sabemos y tenemos en cuenta que se ha convertido en una potente herramienta de trabajo. Pero la realidad es que la adicci{\'o}n a los celulares existe por parte de un gran n{\'u}meros de adolecentes que se pasan el d{\'i}a, literalmente, colgados del celular. Un dato muy interesante y sobre todo es que parece parad{\'o}jico que los tel{\'e}fonos se inventaron para comunicar a las personas entre si y que en nuestro caso sirvan para todo lo contrario: para distraernos y alejarnos de las dem{\'a}s personas.

De todas formas el uso de los celulares debe centrarse en unos criterios consensuados adem{\'a}s debemos reservar un tiempo especifico para comentar en familia como se han utilizado estos dispositivos y como debe ser su uso aparentemente. 

Todos debemos estar informados del uso /abuso que se le da al celular y tomar las medidas que consideren oportunas. Es indispensable que la utilizaci{\'o}n del celular deba ce{\~n}irse a unas reglas claras, asumidas y consensuadas en dado caso no tiene sentido que el celular sirva para aislar a la persona que lo utiliza de la gente que lo rodea.

Las reglas para utilizar el celular son marcadas seg{\'u}n las edades y analizar cu{\'a}les son los momentos o situaciones en los que su uso es adecuado. La tecnolog{\'i}a ha evolucionado en gran medida en la {\'u}ltima d{\'e}cada, y de alguna manera dicha evoluci{\'o}n ha ido afectando a los miembros de nuestra sociedad siendo en mayor proporci{\'o}n la vida diaria de los j{\'o}venes del siglo XXI la que se ve corrompida, y sobre todo los nacidos a partir de 1980.
Esto conlleva un dr{\'a}stico cambio en los h{\'a}bitos de vida que se llevan hoy en d{\'i}a ya que si los compramos con la forma de vida de nuestros antecesores podr{\'i}amos encontrar cambios radicales que en lugar de beneficiarnos nos han ido afectando gravemente, desde el aspecto social hasta la forma que la educaci{\'o}n se ve intervenida, con sus respectivos beneficios y perjuicios. 

El uso de la telefon{\'i}a celular, surgi{\'o} como un medio para facilitar la comunicaci{\'o}n entre personas que se encuentran en diferentes lugares, pero con el paso del tiempo se han presentado nuevos usos.

Esta tecnolog{\'i}a con sus avances, causa en los usuarios una dependencia considerable, lo cual ha hecho que estos sufran cambios, en el entorno social y acad{\'e}mico, por el uso inadecuado de este medio de comunicaci{\'o}n, que d{\'i}a con d{\'i}a avanza m{\'a}s, ofreci{\'e}ndole a sus usuarios no solo poderse comunicar con facilidad, si no tambi{\'e}n le ofrece entretenimiento.

Los usos del tel{\'e}fono m{\'o}vil se han multiplicado, pasando de ser un medio de comunicaci{\'o}n a distancia como lo fuera el tel{\'e}fono de Graham Bell, para ser un aparato que combina dicho uso con el de una agenda electr{\'o}nica, una computadora o hasta un medio de interacci{\'o}n combinando juegos y acceso a la Internet, todo ello aunado a una serie de gadgets adaptados al diminuto aparato, como c{\'a}maras digitales o reproductores de mp3.
Poseer un tel{\'e}fono con tecnolog{\'i}a de punta no es el problema, lo malo est{\'a} en el uso inadecuado que hacen los usuarios por no estar consientes del uso correcto que deben dar a su celular, con lo que se ven afectadas sus relaciones sociales y acad{\'e}micas.

Este tremendo empuje que la nueva tecnolog{\'i}a infringe en nosotros no solamente nos ofrece infinitas posibilidades como medios de acci{\'o}n para resolver problemas de la vida cotidiana como comunicarnos a distancia o resolver tareas, sino que adem{\'a}s ha trasladado nuestra atm{\'o}sfera de desarrollo, que si bien se hab{\'i}a concentrado durante siglos en el entorno natural para luego pasar al entorno urbano alrededor del siglo XVIII, ahora amenaza seriamente en convertirse en un entorno casi totalmente virtual, que implica el desplazamiento de los espacios naturales y urbanos, as{\'i} como el cambio en los h{\'a}bitos sociales y de comportamiento de los individuos, sobre todo de los j{\'o}venes que han nacido y se han desarrollado en la esfera del cambio.

Actualmente internet est{\'a} generando nuevas adicciones, convirti{\'e}ndose en nuevos desaf{\'i}os en el campo de la psiquiatr{\'i}a como el desarrollo de patolog{\'i}as que est{\'a}n asociadas al uso excesivo de esta tecnolog{\'i}a, como: el placer excesivo de estar en l{\'i}nea; irritabilidad o s{\'i}ntomas depresivos al no estar conectados; deterioro de las relaciones familiares y sociales; al igual que negligencia laboral. Con el paso de los a{\~n}os internet se ha vuelto un medio muy popular, y junto con esta creciente notoriedad, apareci{\'o} el uso excesivo y como tal la adicci{\'o}n a internet.

Resulta interesante observar c{\'o}mo los distintos grupos etarios desarrollan diferentes capacidades para adaptarse, cada uno a su manera, a esta evoluci{\'o}n constante.
 Mientras las generaciones m{\'a}s adultas aprendieron a manejar las nuevas tecnolog{\'i}as muy de a poco y fueron tomando de ellas solo lo necesario, los m{\'a}s j{\'o}venes han nacido en la era de la Internet y la telefon{\'i}a celular y, por ende, dependen de ellas para interactuar y comunicarse.

 Por otro lado, en la actualidad los celulares ya no solo se limitan a la funci{\'o}n de comunicar a dos personas como los tel{\'e}fonos de l{\'i}nea para cuyo reemplazo fueron concebidos, sino que han evolucionado hasta transformarse en dispositivos capaces de enviar o recibir im{\'a}genes, videos, documentos, m{\'u}sica; comunicar a m{\'a}s de dos personas a la vez; indicarnos c{\'o}mo llegar a un lugar en fin, se han convertido en peque{\~n}as computadoras port{\'a}tiles.

 El objetivo de este trabajo es determinar si los j{\'o}venes han desarrollado cierto nivel de dependencia de estos nuevos y avanzados dispositivos celulares.

	
	\section{TECNICAS E INSTRUMENTOS}
	
	
Para comenzar con la investigaci{\'o}n se utilizaron las fuentes m{\'a}s viables, que son las personas en si,, al realizarles encuestas, informaci{\'o}n recaudada de art{\'i}culos de revistas y selecci{\'o}n de paginas v{\'i}a internet. 
La muestra fue tomada aleatoriamente por el campus estudiantil de la Universidad Aut{\'o}noma de Baja California, se deambulo para realizar encuestas a personas de diferente edad, g{\'e}nero y miembro del que pertenec{\'i}a como familia; Para as{\'i} poder obtener una muestra con mejores criterios. Despu{\'e}s de haber conocido la respuesta de cada miembro fue traspasado a un graficador, donde se colocaron como a que porcentaje y a que correspond{\'i}a cada una de las preguntas, se adjunto todo para poder obtener una perspectiva mejor, ya que el capturar datos en un archivo para tomar notas y plasmarlas de diferentes formas como graficas, en forma te{\'o}rica y de manera visual, fueron las t{\'e}cnicas que mejor funcionaron. 

	
	
	
	\section{ENCUESTA:}
	
Al llevar a cabo encuestas, consto en entregarle a cada persona una encuesta con once preguntas acerca de c{\'o}mo es la interacci{\'o}n con un m{\'o}vil en el interior de su familia (como influye hacia {\'e}l y hacia ellos). No se trato de alguna manera en particular a las personas en especifico por su g{\'e}nero o edad, simplemente se aclaro que deber{\'i}a de ser un tipo de encuesta aleatoria donde se pudiera obtener una muestra lo m{\'a}s completa posible en tanto a diversidades que pudieran tener correlaci{\'o}n y as{\'i} poder comparar cada una de ellas, ya que fueron trasladadas a graficas en las que detalladamente se da el margen y especificaci{\'o}n que cada pregunta tuvo en particular tuvo su impacto. 
1.	¿C{\'o}mo le ha afectado el uso de dispositivos m{\'o}viles? 

2.	Del 1 al 5, ¿Qu{\'e} tan dependiente es de los dispositivos m{\'o}viles (siendo 1 nada y 5 bastante)?

1.              2.              3.              4.             5.

3.	¿Qu{\'e} miembro de tu familia utiliza m{\'a}s el dispositivo m{\'o}vil?

Padres    Hermanos(a)    Abuelos    Hijos    Otros

4.	¿Desde que edad comenzaste a utilizar dispositivos m{\'o}viles?

0-10          11-20          21-30          31- en adelante

5.	¿Considera que le ha afectado el uso de dispositivos m{\'o}viles a su familia?, si su respuesta es s{\'i} ¿De qu{\'e} manera?

        S{\'i}                                                                              No

6.	¿Considera que el uso de dispositivos m{\'o}viles ha sido positivo o negativo? 

Positivo                    Negativo

7.	¿Te has perdido de momentos familiares por estar utilizando el m{\'o}vil? 

S{\'i}                       No

8.	¿Con que frecuencia ocurre que un familiar se moleste al ser interrumpido mientras utiliza el dispositivo m{\'o}vil? 

Nunca                 Poco                 Bastante

9.	¿Para qu{\'e} utilizas con m{\'a}s frecuencia  el dispositivo?

Comunicaci{\'o}n        Trabajo        Ocio        Escuela        Otros

10.	¿Tiende a refugiarse en el uso de dispositivos m{\'o}viles para evadir la realidad?

S{\'i}                       No

11.	Del 0 al 100, ¿te molesta que te tomen fotograf{\'i}as en las reuniones familiares?

	
	\section{CONCLUSION:}

Al revisar tanto las encuestas como la investigaci{\'o}n realizada, se pudo concluir que los objetivos a los cuales quer{\'i}a llegar fueron alcanzados, ya que al ir realizando encuestas por todo el campus Tijuana de UABC lugar por el cual transitaban personas de diferente g{\'e}nero, edades y miembros de familias. 

Las encuestas tuvieron una amplia gama de muestra poblacional, gracias a ello se pudo deducir que las personas al estar en el m{\'o}vil no interact{\'u}an com{\'u}nmente aun teniendo a alguna persona frente a ellos y que el miembro que sufre m{\'a}s apego a esto son los hijos de familia, j{\'o}venes de edades entre 11 y 20 a{\~n}os, los cuales en mayor cantidad dec{\'i}an que no necesitaban el m{\'o}vil o no depend{\'i}an de {\'e}l, sin embargo, al transitar por el campus se puedo observar a los encuestados utilizando estos m{\'o}viles, lo cual contradice algo que ellos afirmaban, pero, al comenzar con esta encuesta se les comento algunas sugerencias de una manera amable, se volvi{\'o} a analizar a la poblaci{\'o}n de la universidad y pudimos apreciar que en efecto los adolescentes y j{\'o}venes-adultos tienden a usar mucho los dispositivos m{\'o}viles, ya no estaba tan apegado a su m{\'o}vil, eso quiere decir que las sugerencias que le proporcionamos le fueron {\'u}tiles y pudo apreciar que es m{\'a}s agradable interactuar con un miembro de la familia de manera personal y f{\'i}sicamente y as{\'i} tanto dentro de cada familia como de las personas que los rodean. 

	
	
	\section{Referencias:}
	
-(Tecnologica, Grupo Dependencia, 2011)
-(Blogspot, 2010)
-(Olivares Huapaya, 2012)
-(Fandi{\~n}o Leguia, 2015)
-Delia. (2016). Los dispositivos m{\'o}viles como un nuevo integrante en la familia. vestaliaasociados, 2.
-MEPIAR. (2016). Ni{\~n}os y tecnolog{\'i}a: Ventajas e inconvenientes del uso de dispositivos m{\'o}viles. MEPIAR.
-REDACCION. (2014). ¿EL ABUSO DE LOS DISPOSITIVOS M{\'O}VILES EST{\'A} DESTRUYENDO A LAS FAMILIAS? MAMANATURAL , 2.
-VIDA, E. D. (2014). El celular, un obst{\'a}culo para la comunicaci{\'o}n familiar. EL TIEMPO, 2.
-(2015). Dispositivos m{\'o}viles afectan las relaciones sociales. 21 abril del 2016, de Revista Siempre! Presencia de M{\'e}xico Sitio web: http://www.siempre.com.mx/2015/01/dispositivos-moviles-afectan-las-relaciones-sociales/
-ESTILO DE VIDA. (2014). El celular, un obst{\'a}culo para la comunicaci{\'o}n familiar. 21 de abril del 2016, de El tiempo Sitio web: http://www.eltiempo.com/estilo-de-vida/salud/uso-de-celular-afecta-la-comunicacion-familiar/14708103
(2016). El uso del m{\'o}vil es ya uno de los principales conflictos que llevan a las familias a mediaci{\'o}n . 21 abril del 2016, de 20 MINUTOS EDITORA, S.L. Sitio web: http://www.20minutos.es/noticia/2352318/0/uso-movil-principales/conflictos-familias/mediacion-adolescentes/
-Teresa Arnaboldi. (2014). suelta ya ese celular . 21 abril del 2016, de hacer familia 
-(Emprendices, 2015; Dirigentesdigital.com, 2014; yeeply, 2012-2016; Educativa, 2016)
-(Martinez Ruvalcaba, Enciso Arambula, Y Gonzalez Castillo, 2015)
-(Inarejos Merino)
-(Solucionesc2, 2015)
-(Vargas Ramirez, 2016)
\end{document}
